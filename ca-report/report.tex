\documentclass[12pt,fullpage]{article}
\usepackage[american]{babel}
\usepackage{csquotes}
\usepackage[style=apa,backend=biber]{biblatex}
\DeclareLanguageMapping{american}{american-apa}
\bibliography{report.bib}
\linespread{1.5}


\begin{document}
\title{A Web-Based Asynchronous System in F\#}
\author{Chua Jie Sheng}
\maketitle

\section{Project Objective}
The objective of this project is to explore the use of a high-level functional
language for web development. Currently web system is organized in three tiers,
namely web interface, server and databases. Each of this component runs on a
separate computer. Implementing a web application would thus require one to master a
minimum of three different languages. Using multiple languages to implement a
single application could easily result in impedance mismatch problem when the data
types are not strictly matched \parencite{links06}. In addition, web programming
has been constrained by the protocol, standard and browser implementation.
This restricts the evolution of web technology beside to higher adoption of
JavaScript implementation \parencite{balat09}. We explore using a high-level
functional language to allow the implementation to be accomplished using a single language,
which in turn could allow us to ensure a higher level of correctness in the implementation
via type checking provided by functional languages.

\section{Literature Review}
\subsection{Introduction}
In the recent years, the web has been moving towards dynamic and interactive
interaction from the usual data-centric approach, while the methods to create web
application remains largely the same. There have been many attempt to adopt functional
languages for web programming. However, most see limited adoption. Among these attempts are
Hop \parencite{serrano07}, Links \parencite{links06}, Ocsigen \parencite{balat06},
Pit \parencite{paper4}, SML.NET \parencite{benton04}.

\subsection{Form-based Interaction}
In web application, using forms to request information from users is a trivial
task. Client-side developer develops the user interface containing the form widget
while server-side developer develops the services to receive the information.
More often than not, forms developed by client-side developer do not present
all information required by the services; this result in a problem known as
impedance mismatch \parencite{links06}.
\\
In addition, with the extensive use of cache, certain information may have
been cached on the web browser. Leveraging on such capabilities would make it
possible for the form to varies the input parameters and leverage on the cached
values. This could result in fewer input parameters and thus better
user interaction experience.
\\
As web application is constrained by the technology and the implementation of web
browser and web server, the use of HTTP request method is restricted to one per
form\footnote{With the exception that web developers explicitly design the
page to handle individual form submission manually with different or multiple
Ajax calls.}. This results in restriction in the types of data sent to the server.
When GET request is used, the information sent is explicitly shown to users which could
register a bookmark to return to the same page. However, this is different from POST
request, where data sent is not shown to users and a bookmark to a POST
services bring users to a dead page \parencite{balat06}.

\subsection{Box Notion}
Currently, with the implementation of Box notion in Ocsigen, web developers
could declare fragment of HTML codes which could be reused. However, this restricted
to only predefined HTML fragment and not the content within the HTML. Alternatively in
Ocsimore, the use of box notion allows the use of special syntax to reuse
previously declared content with finer control \parencite{balat09}.
\\ Box 1:
\begin{verbatim}
The text at the end is in bold: **<<content>>**.
\end{verbatim}
Box 2:
\begin{verbatim}
Let us call 1: <<wikibox box='1' | In bold >>
\end{verbatim}
The result of the second box would be:
\\
\centerline{``Let us call 1: The text at the end is in bold: {\bf In Bold}.''}
\\
This wiki-style allows web developers to reuse previously declared HTML
content more rapidly, without having to explicitly modify the HTML content.

\subsection{Automated User Interface Testing}
The inspiration from the existing testing framework \parencite{nagarani12}, has
given rise to an existing need to continuously testing every part of the application to
prevent regression \parencite{collins12} from entering the system. This
consequently leads to a need for an automated GUI testing framework that could
test the application as a whole before deployment.
\\
Furthermore, automated testing helps to ensure that any new code added to the
application works compatibly with other existing functionality. In the event
where the addition of new code causes the system to break down, the programmer
should be notified so that the problem could be rectified before the system
becomes production.
\\
Scripted automated testing also allow the possibility of continuous integration
testing. Continuous integration testing tests for issues with the system after
integrating a new set of codes. Continuous integration thus allows these issues
to be discovered as early as possible, thus reducing the testing cost
\parencite{liu09}.

\subsection{Integrating Web Services}
When the web is made up of multiple services provided by various providers,
there is eventually a need to integrate different services so as to provide a
complete solution for the consumers. Thus, it is important to provide a mechanism
such that other services or applications could inter-operate with the services the
application is providing \parencite{bosworth01}. This could be in the form of
Extensible Markup Language (XML) or JavaScript Object Notion (JSON)
\parencite{lin12}. Using JSON, web developers could simply consume the web
service by obtaining the JSON objects and parsing the data accordingly. This allows
more information to be readily available without web developers having to
request from the server.

\subsection{Conclusion}
Strongly typed web system is still relatively new and there exists a substantial
room for improvement in many areas such as libraries and testing frameworks. In
order to increase the adoption of a given platform, the platform needs to
integrated with other platform to consume and provide services. This
collaboration is of utmost important in a cooperative environment.

\section{Progress}
\subsection{Lack of Support in F\#}
This project originally looks into implementing web application in F\# with Pit
\footnote{Pit is a F\# to JavaScript compiler.} \parencite{giannini12}. However,
due to the poor popularity and the lack of support for the project, Pit project eventually
died down and the project site\footnote{The project site was at pitfw.org.} was
closed down. Thus, we began to look for an alternative functional web framework and
eventually decided to work on the project using Ocsigen framework and OCaml
language. Ocaml, being a more mature lanaguge, provides greater support thanks
to its larger use base and continuous support from INRIA. Nevertheless, besides
the above-mentioned improvement, we would like to present additional enhancement
we have gathered from the experience we have gained when implemented a graffiti application using
Eliom application server under Ocsigen. In the application, the graffiti allows players to draw
shapes onto a HTML canvas which could be used to interact with other players
viewing the same page.

\subsection{Graffiti Web Application}
This graffiti web application was part of the Ocsigen platform tutorial. However, due
to the complexity of the platform coupled with the fact that it is mainly
maintained as a research project, most of the documentations, tutorials and
examples have been outdated\footnote{I had sent in my graffiti application as a patch
to update the existing tutorial.}. Therefore, a huge amount of time was devoted
in the developing the examples from scratch.
\\
Eliom, as a platform has a substantially rich set of libraries which are readily
available for any new developer to adopt \parencite{eliom}. However, despite of
the great advantage, more improvements could be done to make Eliom even a better
choice as compared to others. This includes
various enhancements in the library support for synchronizing server and client-side variables,
libraries to ensure strongly-typed formed based interaction, automated
integrated testing and other areas mentioned in the previous section.

\subsection{Synchronizing of Variables}
Currently, in Eliom, synchronizing server-side and client-side variable require
explicit server function declaration such as:
\begin{verbatim}
let rpc_put_key = server_function Json.t<int>
  (fun key -> key := new_key; Lwt.return())
\end{verbatim}
This function explicitly updates the key value on the server side. On the other
hand, getting the key value from the server from the client require another function
such as:
\begin{verbatim}
let rpc_get_key = server_function Json.t<unit>
  (fun () -> Lwt.return key)
\end{verbatim}
As the number of variables shared increases, the number of functions needed to
synchronize the variables will increase twice\footnote{For every variable,
one function is needed to update the server-side variable while another to
retrieve the variable.} the speed. A simple generic function which could
be used to update and retrieves the variable would be helpful in reducing
the duplication of code over the application.
\\
Using the understanding of the synchronizing mechanism, we could adopt this
in providing a dynamic form-based interaction. This interaction technique
would allow the form to be updated with the correct values when the users
modify some existing properties.

\section{Research Plan}
For the remaining months, I intend to implement a seminar web application with
a initial focus on providing a strongly-typed form-based interaction method and
library. Using this library, the objective is to create strongly-typed forms
and provide the developers with the tool which enables them to maintain the type
safety.
\\
Furthermore, with the experience acquired from the implementation of a
form-interaction library, I intend to implement the box notion mentioned in the
previous section. The use of the box notion in the seminar web application is to
encourage the dynamic reuse of code fragment throughout the code base. In
addition, in the process of implementing the seminar web application, I intent
to adopt a test oriented design methodology in which I aim to achieve high
testing metrics for the overall code base.
\\
This implementation strategy aims to prefect my skill in unit testing so as to
implement a method to test the web application as a whole. Using the experience
obtained from unit testing, I wish to develop methods to validate HTTP response
and HTTP responses pages in the future. A good method to validate the response
pages would first retrieve the pages, then parsing the pages into OCaml data
types which could be used to validate the response's page structure, types and
values. This method could also be used to test the system as a whole including
the database.

\printbibliography
\end{document}
