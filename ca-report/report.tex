\documentclass[12pt,fullpage]{article}
\usepackage[american]{babel}
\usepackage{csquotes}
\usepackage[citestyle=apa,backend=bibtex]{biblatex}
\DeclareLanguageMapping{american}{american-apa}
\bibliography{report.bib}
\linespread{1.5}

\begin{document}
\title{A Web-Based Asynchronous System in F\#}
\author{Chua Jie Sheng}
\maketitle

\section{Project Objectives}
The objective of this project is to explore to use of a high-level functional
language for web development. Currently web system is organized in three tiers,
namely web interface, server and databases. Each of this component runs on a
separate computer. Implementing a web application would thus need to master a
minimum of three different languages. Using multiple language to implement a
single platform would easily result in impedance mismatch problem when the data
types are not strictly matched \parencite{links-fmco06}. In addition, web programming
have been constrained by the protocols, standards and browsers implementation.
This restricts the evolution of web technology beside higher adoption of
JavaScript implementation \parencite{p311-balat}. We explore using a high-level
functional language to allow the implementation to exist on a single language,
which in turns allow us to ensure strong correctness of implementation
via type checking provided by functional languages.

\section{Literature Review}
\subsection{Introduction}
In the recently years, the web have been moving towards dynamic and interactive
interaction instead of being data-centric, while the methods to create web
application remains vastly the same. There have many attempt to adopt functional
language for web programming but most see limited adoption. These include
Hop \parencite{p1001-serran}, Links \parencite{links-fmco06}, Ocsigen \parencite{p84-balat},
Pit \parencite{paper_4}, SML.NET \parencite{p53-Benton}.

\subsection{Form-based Interaction}
In web application, using forms to request information from users is a trivial
task, client-side developer develops the user interface contain the form widget
while server-side developer develops the services to receive the information.
More often than not, forms developed by client-side developer does not present
all information required by the services, this result in a problem called as
impedance mismatch \parencite{links-fmco06}.
\\
As web application are constrained by the technology and implementation of web
browser and web server, the use of HTTP request method are restricted to one per
form\footnote{With the exception that web developer explicitly design the
page to handle individual form submission manually with different or multiple
Ajax calls.}. This result in restriction in the types of data sent to the server.
When using GET, the information sent is explicitly shown to the user which could
register a bookmark to return to the same page. But this is different from POST
request, where data send is not shown to the user and a bookmark to a POST
services bring you to a dead page \parecite{p84-balat}.

\subsection{Box Notion}


\subsection{Testing of Web Services}

\subsection{}
\subsection{Conclusion}
1. sending both get and post method in the same form p84-balat


\section{Progress}
\subsection{Graffiti Web Application}

\subsection{Form-based Interaction}
\section{Research Plan}
\section{References}
\printbibliography
\end{document}
