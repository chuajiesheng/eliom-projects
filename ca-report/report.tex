\documentclass[12pt,fullpage]{article}
\usepackage[american]{babel}
\usepackage{csquotes}
\usepackage[style=apa,backend=biber]{biblatex}
\DeclareLanguageMapping{american}{american-apa}
\bibliography{report.bib}
\linespread{1.5}


\begin{document}
\title{A Web-Based Asynchronous System in F\#}
\author{Chua Jie Sheng}
\maketitle

\section{Project Objectives}
The objective of this project is to explore to use of a high-level functional
language for web development. Currently web system is organized in three tiers,
namely web interface, server and databases. Each of this component runs on a
separate computer. Implementing a web application would thus need to master a
minimum of three different languages. Using multiple language to implement a
single platform would easily result in impedance mismatch problem when the data
types are not strictly matched \parencite{links06}. In addition, web programming
have been constrained by the protocols, standards and browsers implementation.
This restricts the evolution of web technology beside higher adoption of
JavaScript implementation \parencite{balat09}. We explore using a high-level
functional language to allow the implementation to exist on a single language,
which in turns allow us to ensure strong correctness of implementation
via type checking provided by functional languages.

\section{Literature Review}
\subsection{Introduction}
In the recently years, the web have been moving towards dynamic and interactive
interaction instead of being data-centric, while the methods to create web
application remains vastly the same. There have many attempt to adopt functional
language for web programming but most see limited adoption. These include
Hop \parencite{serrano07}, Links \parencite{links06}, Ocsigen \parencite{balat06},
Pit \parencite{paper4}, SML.NET \parencite{benton04}.

\subsection{Form-based Interaction}
In web application, using forms to request information from users is a trivial
task, client-side developer develops the user interface contain the form widget
while server-side developer develops the services to receive the information.
More often than not, forms developed by client-side developer does not present
all information required by the services, this result in a problem called as
impedance mismatch \parencite{links06}.
\\
As web application are constrained by the technology and implementation of web
browser and web server, the use of HTTP request method are restricted to one per
form\footnote{With the exception that web developer explicitly design the
page to handle individual form submission manually with different or multiple
Ajax calls.}. This result in restriction in the types of data sent to the server.
When using GET, the information sent is explicitly shown to the user which could
register a bookmark to return to the same page. But this is different from POST
request, where data send is not shown to the user and a bookmark to a POST
services bring you to a dead page \parencite{balat06}.

\subsection{Box Notion}
Currently implemented in Ocsigen, web developer could declare fragment of HTML
codes which could be reused. But this restricted to predefined HTML fragment
but not content within the HTML. Alternatively in Ocsimore, the use of box
notion allows the uses of special syntax to reuse previously declared content,
with finer control \parencite{balat09}.
\\* Box 1:
\begin{verbatim}
The text at the end is in bold: **<<content>>**.
\end{verbatim}
Box 2:
\begin{verbatim}
Let us call 1: <<wikibox box='1' | In bold >>
\end{verbatim}
The result of the second box would be:
\\*
\centerline{``Let us call 1: The text at the end is in bold: {\bf In Bold}.''}
\\*
This wiki-style allow web developer to reuse previously declared HTML
content quicker. Without having to explicitly modified HTML content.

\subsection{Integrating Web Services}
The web become made up of multiple service provided by various providers,
there is eventually a need to integrate different services so as to provide a
complete solution for the consumers. Thus it is important to provide a mechanism
such that other service or application could inter-operate with the services the
application is providing \parencite{bosworth01}. This could be in the form of
Extensible Markup Language (XML) or JavaScript Object Notion (JSON)
\parencite{lin12}. Using JSON, web developer could simply consume the web
service by obtain the JSON objects and parse the data accordingly. This allow
more information to be readily available instead of having to request from
the server.

%% \subsection{Testing of Web Services}

\subsection{Conclusion}
Strongly typed web system is still relatively new and there exist a substantial
amount of improvement to be done. These include area such as libraries, and also
testing framework. In order for wide adoption of a given platform ability to
integrate with other platform of services is also key in a cooperative
environment.

\section{Progress}
\subsection{Lack of Support in F\#}
This project originally look into implementing web application in F\# with Pit
\footnote{Pit is a F\# to JavaScript compiler.} \parencite{giannini12}. But due
to lack of support and contributor for the project, the project eventually die
down and the project site\footnote{The project site was at pitfw.org.} was
shutdown. We begin looking for alternative to a functional web framework and
eventually decide to work on the project using Ocsigen framework using OCaml
language. OCaml being a more mature provide more support due to better
maintenance. Nevertheless, together the above mention improvement, we would
mention addition improvement from the experience we gain while implementing
a sample application. The graffiti allows players to draw onto onto the HTML
canvas while others play forms updates with the new shapes.

\subsection{Graffiti Web Application}

\subsection{Synchronizing of Variables}
Currently in Eliom, synchronizing server-side and client-side variable require
explicit server function declaration such as:
\begin{verbatim}
let rpc_put_key = server_function Json.t<int>
  (fun key -> key := new_key; Lwt.return())
\end{verbatim}
This function explicitly updates the key value on the server side. While to
get the key value from the server from the client require another function
such as:
\begin{verbatim}
let rpc_get_key = server_function Json.t<unit>
  (fun () -> Lwt.return key)
\end{verbatim}
As the number of variable shared increase, the number of function needed to
synchronize the variable will increase twice\footnote{For every variable,
one function is needed to update the server-side variable while another to
retrieve the variable.} the speed. A simple generic function which could
be use to update and retrieve the variable would be helpful in reducing
the duplication of code over the application.


\subsection{Form-based Interaction}
\section{Research Plan}

\printbibliography
\end{document}
